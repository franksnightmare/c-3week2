\documentclass[11pt]{article}

\usepackage{times}
\usepackage[english]{babel}

% -----------------------------------------------
% especially use this for you code
% -----------------------------------------------

\usepackage{courier}
\usepackage{listings}
\usepackage{color}
\usepackage{tabularx}
\usepackage{graphicx}

\definecolor{Gray}{gray}{0.95}

\definecolor{mygreen}{rgb}{0,0.6,0}
\definecolor{mygray}{rgb}{0.5,0.5,0.5}
\definecolor{mymauve}{rgb}{0.58,0,0.82}

\lstset{language=C++,
	basicstyle = \normalsize\ttfamily,   % the size and fonts that are used
	tabsize = 2,                    % sets default tabsize
	breaklines = true,              % sets automatic line breaking
	keywordstyle=\color{blue}\ttfamily,
	stringstyle=\color{red}\ttfamily,
	commentstyle=\color{mygreen}\ttfamily,
	numbers=left,
	keepspaces=true,
	showspaces=false,
	showstringspaces=false,
}

\begin{document}

\title{Programming in C/C++ \\
       Exercises set two: advanced class templates
}
\date{\today}
\author{Christiaan Steenkist \\
Jaime Betancor Valado \\
Remco Bos \\
}

\maketitle
\section*{Exercise 9, Needle fishing}
We made a function that returns the place of the first template class in a haystack of classes.

\subsection*{Code listings}
\lstinputlisting[caption = main.cc]{src/a9/main.cc}
\lstinputlisting[caption = type.h]{src/a9/type.h}

\section*{Exercise 10, Needle fishing with nested class}
We changed exercise 9, such that it now uses a nested helper class

\subsection*{Code listings}
\lstinputlisting[caption = main.cc]{src/a10/main.cc}
\lstinputlisting[caption = type.h]{src/a10/type.h}

\section*{Exercise 13, Binary operators}
We made a class that overloads binary operators

\subsection*{Code listings}
\lstinputlisting[caption = main.cc]{src/a13/main.cc}
\lstinputlisting[caption = main.ih]{src/a13/main.ih}
\lstinputlisting[caption = adder.add.cc]{src/a13/adder.add.cc}
\lstinputlisting[caption = adder.h]{src/a13/adder.h}
\lstinputlisting[caption = adder.ih]{src/a13/adder.ih}
\lstinputlisting[caption = adder.value.cc]{src/a13/adder.value.cc}
\lstinputlisting[caption = arithmetic.h]{src/a13/arithmetic.h}
\lstinputlisting[caption = arithmetic.ih]{src/a13/arithmetic.ih}
\lstinputlisting[caption = binopsbase.h]{src/a13/binopsbase.h}
\lstinputlisting[caption = binopsbase.ih]{src/a13/binopsbase.ih}

\section*{Exercise 14, Generic variadic template }
We changed the class BinopsBase to a variadic template class using a set of int argument.

\subsection*{Code listings}
\lstinputlisting[caption = main.cc]{src/a14/main.cc}
\lstinputlisting[caption = main.ih]{src/a14/main.ih}
\lstinputlisting[caption = operations.h]{src/a14/operations.h}
\lstinputlisting[caption = operations.ih]{src/a14/operations.ih}
\lstinputlisting[caption = binopsbase.h]{src/a14/binopsbase.h}
\lstinputlisting[caption = binopsbase.ih]{src/a14/binopsbase.ih}

\end{document}